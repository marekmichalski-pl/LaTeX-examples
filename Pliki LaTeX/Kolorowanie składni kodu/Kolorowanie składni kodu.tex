\documentclass[11pt,a4paper]{article}   
\usepackage[polish]{babel}
\usepackage[utf8]{inputenc}
        
 
\usepackage{listings}   %paczka listingow programow
\usepackage{color}

%definiuje kolory dla skladni
\definecolor{bluekeywords}{rgb}{0.13,0.13,1}
\definecolor{greencomments}{rgb}{0,0.5,0}
\definecolor{redstrings}{rgb}{0.9,0,0}

% okresla tytul dokumnetu
\title{Kolorowanie składni kodu.}                                                           
\author{}                   		 	                    % autor
\date{}                                     		        % data wydania 


\lstset{language=[Sharp]C,
showspaces=false,
showtabs=false,
breaklines=true,
numbers=left,
stepnumber=1,
showstringspaces=false,
breakatwhitespace=true,
escapeinside={(*@}{@*)},
commentstyle=\color{greencomments},
keywordstyle=\color{bluekeywords}\bfseries,
stringstyle=\color{redstrings},
basicstyle=\ttfamily
}


\begin{document}

\maketitle  % wstawia tytul

 % implementacja  singletona pobrana z http://www.altcontroldelete.pl/artykuly/konstrukcyjny-wzorzec-projektowy-singleton-implementacja-w-c-/
\begin{lstlisting}
public sealed class Singleton    
{
    private static Singleton m_oInstance = null;
    private int m_nCounter = 0;
 
    public static Singleton Instance
    {
        get
        {
            if (m_oInstance == null)
            {
                m_oInstance = new Singleton();
            }
            return m_oInstance;
        }
    }
 
    public void DoSomething()
    {
        Console.WriteLine("Hello World po raz {0}!", m_nCounter++);
    }
 
    private Singleton()
    {
        m_nCounter = 1;
    }
}
    \end{lstlisting}
\end{document}   