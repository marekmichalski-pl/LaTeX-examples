\documentclass[11pt,a4paper]{article}       % okreslenie klasy dokumentu, rozmiaru czcionki bazowej, formatu
\usepackage[polish]{babel}                  % dokument w jezyku polskim
\usepackage[utf8]{inputenc}                 % system kodowania
\usepackage{bchart}                         % dodanie pakietu odpowiedzialnego za generowanie wykresow
\usepackage{gensymb}                        % dodanie pakietu odpowiedzialnego za symbole (w przykladzie symbol stC)

% okresla tytul dokumnetu
\title{Generowanie wykresu.}                                                           
\author{}                                   % autor
\date{}                                     % data wydania (pzostawiono puste, nie bedzie wyswietlona)

\begin{document}
\maketitle

\begin{bchart}[step=5, min=15, max=45, width=11cm,unit=\degree C]   %kod tworzacy wykres (definicja punktow)
\bcbar[label=09:08]{37.38}
\smallskip
\bcbar[label=09:03]{26.97}
\smallskip
\bcbar[label=08:58]{30.67}
\smallskip
\bcbar[label=08:53]{39.38}
\smallskip
\bcbar[label=08:48]{36.68}
\end{bchart}

\end{document}
