\documentclass[11pt,a4paper]{article}       % okreslenie klasy dokumentu, rozmiaru czcionki bazowej, formatu
\usepackage[polish]{babel}                  % dokument w jezyku polskim
\usepackage[utf8]{inputenc}                 % system kodowania
\usepackage{enumerate}

% okresla tytul dokumnetu
\title{Listy i wyliczenia.}                                                           
\author{}                                   % autor
\date{}                                     % data wydania (pzostawiono puste, nie bedzie wyswietlona)

\begin{document}
\maketitle


Lista pierwsza:
 \begin{enumerate}
\item Pierwszy element
\item Drugi  element
\item Trzeci element
\item Czwarty element
\item \ldots
\end{enumerate}

\vspace{10mm}		%odstep 10 mm
Lista druga: 
 \begin{enumerate}[I.]
\item Pierwszy element
\item Drugi  element
\item Trzeci element
\item Czwarty element
\item \ldots
\end{enumerate}

\vspace{10mm}		%odstep 10 mm
Lista trzecia:
\begin{itemize}
\item Pierwszy element
\item Drugi  element
\item Trzeci element
\item Czwarty element
\item \ldots
\end{itemize}

\end{document}
